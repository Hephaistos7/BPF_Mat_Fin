% ======Commands==========


% Vecteurs
\newcommand{\vect}[1][]{%  Vecteur AB
  \def\DefaultA{#1}%
  \vectNext
}
\newcommand{\vectNext}[2][]{%
%  DefaultParam 1: \DefaultA
%  DefaultParam 2: #1
%  Param 3: #2
 {\overrightarrow{#2}%
  \ifthenelse{\equal{\DefaultA}{\empty}}{}%
  {\left(\begin{array}{c} \DefaultA \\ #1 \end{array}\right)}%
  }
} 

\newcommand\co[2] % coordonnées des vecteurs
{\left(\begin{array}{c} #1 \\ #2 \end{array}\right)}



%Symbols mathématiques
\renewcommand{\leq}{\leqslant}
\renewcommand{\geq}{\geqslant}

\newcommand{\lk}{\left(}
\newcommand{\rk}{\right)}

\renewcommand{\degre}{{\ensuremath ^\circ}}

\DeclareMathOperator{\sgn}{sgn}

\newcommand{\ca}
{\begin{center}\emph{Calculatrice autorisée, toutes les étapes de calcul doivent être explicitées!}\end{center}}

\newcommand{\cna}
{\begin{center}\emph{L'usage de la calculatrice n'est pas permis!}\end{center}}

\newcommand{\rep}{\left(O;\vec{\imath},\vec{\jmath}\right)}
\newcommand{\repp}{\left(O;\vec{\imath},\vec{\jmath},\vec{k}\right)}

%Arc
\newcommand{\arc}[1]{%
    \raisebox{-3.3pt}{%
   \begin{tikzpicture}%
       \node at (0,0) {$#1$};%
      \draw[yshift=-0.2cm,xshift=-0.04cm] (45:0.5cm) arc (45:135:0.4cm);%
   \end{tikzpicture}%
   }%
}



%Ensembles
\newcommand{\C}{\mathbb{C}}
\newcommand{\R}{\mathbb{R}}
\newcommand{\Q}{\mathbb{Q}}
\newcommand{\D}{\mathbb{D}}
\newcommand{\Z}{\mathbb{Z}}
\newcommand{\N}{\mathbb{N}}

%---------------------- weisse Feil an enger faarweger Kescht
\newcommand{\arrow}[1][1]
 {\begin{tikzpicture}[rounded corners=0,scale=#1, x=.33em,y=.33em]
    \path[right color=\scriptcolor, left color=white] (0,0)--(4.33,0)--(4.33,2)--(0,2)--cycle;
    \path[fill=white] (1,0.66)--(3,0.66)
                              --(2.33,-0.066)
                              --(3.33,-0.066)
                              --(4.33,1)
                              --(3.33,2.066)
                              --(2.33,2.066)
                              --(3,1.33)
                              --(1,1.33)
                              --cycle;
  \end{tikzpicture}
   }

%Attention-Bild
\newcommand{\attention}[1][1]
 {\begin{tikzpicture}[scale=#1, x=1.25em, y=1.25em,rounded corners=0]
    \path[shading = radial, inner color=newdarkred!50, outer color=newdarkred] (0,0)--(1,0)--(0.5,0.85)--cycle;
    \draw[white, fill=white] (0.48,0.25) -- (0.43,0.58) arc (190:-10:0.07) --
                             (0.57,0.58) -- (0.52,0.25) -- cycle;
    \draw[fill, white] (0.5,0.12) circle (0.05);
  \end{tikzpicture}
   }
   
% Haisecher wéi am Heft 
\newcommand{\heft}[2]
  {\noindent
   \begin{tikzpicture}[scale=1.05, line cap=round,line join=round,>=triangle 45,x=1.0cm,y=1.0cm]
   \draw [color=gro,dash pattern=on 1pt off 1pt, xstep=0.5cm,ystep=0.5cm] (0,0) grid (#1,#2);
   \clip(0,0) rectangle (#1,#2);
   \end{tikzpicture}
   }

% ligne vun Haisecher (Jeff)
\newcommand{\blank}[1]{{\heft{#1}{0.5}}}
\newcommand{\midheft}[2]{{\raisebox{-0.15cm}{\heft{#1}{#2}}}}
\newcommand{\midblank}[1]{{\raisebox{-0.15cm}{\heft{#1}{0.5}}}}
\newcommand{\blankfield}[1]{{\centering{\heft{15}{#1}}}}
\newcommand{\halfblankfield}[1]{{\centering{\heft{7}{#1}}}}
\newcommand{\blankspace}{\ \midblank{0.5}}
   
%Underline mat g duerchstreichen
\newcommand{\uline}[1]{\underline{\smash{#1}}}

%Commentaires, question , réponse
\newboolean{comment}
\setboolean{comment}{true}
%---------------------- Comment
\newcommand\id[1]{\ifthenelse{\boolean{comment}}{{\color{lightblue}\slshape #1}}}{}
%---------------------- Question
\newcommand{\que}[1]
{\ifthenelse{\boolean{comment}}
 {{\color{red}\par\noindent\textbf{Q: }#1\color{black}
   }
  }
 {}
}
%---------------------- Answer
\newcommand{\ans}[1]
{\ifthenelse{\boolean{comment}}
 {{\color{darkgreen}%
% \par\noindent%
  \textbf{R: }#1%
  \color{black}}
  }
 {}
} 
\newcommand{\on}{\setboolean{comment}{true}} % unschalten
\newcommand{\off}{\setboolean{comment}{false}} %auschalten

\definecolor{lightblue}{RGB}{0,75,204} % bloschreiwen
\newcommand\idon{\color{lightblue}\slshape\noindent}
\newcommand\idoff{\color{black}\normalfont}

% Entete Examen
\newcommand{\entete}[4] % Session, Section , Math I ou Math II, Durée
{\renewcommand{\arraystretch}{1.2}
\begin{center}
\begin{tabular}{|c|p{1cm}cp{1cm}|p{1.5cm}cp{.5cm}p{0pt}|}
  \hline
  Code branche &
    \multicolumn{6}{c}{Ministère de l'Éducation nationale, de l'Enfance et de la Jeunesse}& \\
  MATHE #3 & \multicolumn{6}{c}{\scshape Examen de fin d'études secondaires techniques}& \\
  & \multicolumn{6}{c}{Régime technique - Session #1}& \\
  \hline
  \textbf{Épreuve} && Branche &&& Division/Section && \\
  \textbf{écrite} &&&&&&& \\
  \cline{1-1}
  Durée de l'épreuve && \scshape\Large Mathématiques #3 &&& \scshape\Large #2 && \\
  #4 &&&&&&& \\
  \cline{1-1}
  Date de l'épreuve &&&&&&& \\
  &&&&&&& \\
  \hline
\end{tabular}
\end{center}
\renewcommand{\arraystretch}{1}
}

\newcommand{\dbox}
{\begin{minipage}{0.40\linewidth} % 
\begin{framed}
\vspace{0.45cm}
Nom: \dotfill\\\\
Prénom: \dotfill \\\\
Classe :\dotfill \\\\
Date:\dotfill
\end{framed}
\end{minipage}\hfill
\begin{minipage}{0.29\linewidth}
\begin{center}
Volet statistique\\
\begin{TAB}(30pt,24pt)[2pt,3.9cm,3.9cm]{|c|c|}{|c|c|c|c|c|c|c|}
50-60 &\\40-49 &\\ 30-39 &\\ 29-20&\\ 19-10 &\\ 01-09 &\\ Moyenne & \\ 
\end{TAB}
\end{center}
\end{minipage}\hfill
\begin{minipage}{0.29\linewidth}
\begin{framed}
\textbf{Note:}\\
\vspace{1.15cm}
\end{framed}
Signature des parents:\\\\
.\dotfill
\end{minipage}\\
\vspace{0.5cm}\\
Remarques:\dotfill \\\\.\dotfill\\[2ex]
}

%Questions
\newcounter{temp}
\newcommand{\quest}[1] % Question environnement
{\refstepcounter{temp}
\bigskip
 {\bfseries\large Question \arabic{temp}} \hfill {(#1 points)} \\[1.5ex] 
}

\newcounter{templ}
\newcommand{\ex} % Question environnement
{\refstepcounter{templ}
\bigskip
 {\bfseries\large Exercice \arabic{templ}}\\} %[1.5ex] 


%------------------------------- Jeff

%-------------------- Operators

\DeclareMathOperator{\Div}{div}
\DeclareMathOperator{\mil}{mil}

%-------------------- Tafelrechnung


%-------------------- TikZ

%Balance for visualizing equations
\newcommand{\equationBalance}[2]{

	\begin{tikzpicture}[
	rounded corners=0,
	joint/.style = { inner sep=0, minimum size=0.1mm,fill=black,draw,circle},
	arg/.style = { inner sep=0,rectangle}
	]
	\node[joint] (base) at (0,0) {};
	\node[joint,below left= 0.8 of base] (base1) {};
	\node[joint,below right= 0.8 of base] (base2) {};
	\draw[fill=black!20] (base.center) -- (base1.center) -- (base2.center) -- cycle;
	
	\node[joint,left= 1.5 of base.north] (leftJoint) {};
	\node[joint,right= 1.5 of base.north] (rightJoint) {};
	\draw (leftJoint) -- (rightJoint);
	
	%right side
	\node[joint,above left= 0.8 of leftJoint](leftBase) {};
	\draw (leftBase) -- (leftJoint);
	
	\node[joint,left= 0.8 of leftBase] (leftBaseLeftJoint) {};
	\node[joint,right= 0.8 of leftBase] (leftBaseRightJoint) {};
	\draw (leftBaseLeftJoint) -- (leftBaseRightJoint);
	
	\node[joint,above left= 0.5 of leftBaseLeftJoint] (leftBaseLeftArm) {};
	\draw (leftBaseLeftJoint) -- (leftBaseLeftArm);
	\node[joint,above right= 0.5 of leftBaseRightJoint] (leftBaseRightArm) {};
	\draw (leftBaseRightJoint) -- (leftBaseRightArm);
	
	%left side
	\node[joint,above right= 0.8 of rightJoint](rightBase) {};
	\draw (rightBase) -- (rightJoint);
	
	\node[joint,left= 0.8 of rightBase] (rightBaseLeftJoint) {};
	\node[joint,right= 0.8 of rightBase] (rightBaseRightJoint) {};
	\draw (rightBaseLeftJoint) -- (rightBaseRightJoint);
	
	\node[joint,above left= 0.5 of rightBaseLeftJoint] (rightBaseLeftArm) {};
	\draw (rightBaseLeftJoint) -- (rightBaseLeftArm);
	\node[joint,above right= 0.5 of rightBaseRightJoint] (rightBaseRightArm) {};
	\draw (rightBaseRightJoint) -- (rightBaseRightArm);
	
	
	\newsavebox\leftbox
	
	\begin{lrbox}{\leftbox}
		#1
	\end{lrbox}
	
	\newsavebox\rightbox
	
	\begin{lrbox}{\rightbox}
		#2
	\end{lrbox}
	
	\node[above=0mm of leftBase,arg](leftArg) {\usebox\leftbox};
	\node[above=0mm of rightBase,arg](rightArg) {\usebox\rightbox};
	
	\global\let\leftbox\relax
	\global\let\rightbox\relax

	\end{tikzpicture}

}




%------------------ tableau de proportionnalité

\newenvironmentx{proptableau}[3][1=3.5cm,2=5,3=1cm,usedefault]
{\setstretch{1}%\setlength{\extrarowheight}{10pt}
	\begin{center}
		\begin{tabular}{|p{#1}|*{#2}{>{\centering\arraybackslash}p{#3}|}}
}{
\end{tabular}\end{center}}





% ----------------------- axis
\pgfplotsset{
	standardaxis/.style={
		enlarge x limits=0.15,
		enlarge y limits=0.15,
		%every axis x label/.style={at={(current axis.right of origin)},anchor=north west},
		every axis y label/.style={at={(current axis.above origin)},anchor=north east},
		axis lines=middle,
		axis line style={-Stealth,very thick},
		xmin=-10.5,xmax=10.5,ymin=-10.5,ymax=10.5,
		xtick distance=1,
		ytick distance=1,
		extra x ticks={0},
		extra x tick style={grid style={black},xticklabel style={below left=-0.2em},xtick style={black}},
		xticklabel style={font=\tiny},
		yticklabel style={font=\tiny},
		xlabel=$x$,
		ylabel=$y$,
		grid=major,
		grid style={thin,densely dotted,black!20}
}}

\pgfplotsset{
	standardaxis2/.style={
		enlarge x limits=0.15,
		enlarge y limits=0.15,
		%every axis x label/.style={at={(current axis.right of origin)},anchor=north west},
		every axis y label/.style={at={(current axis.above origin)},anchor=north east},
		axis lines=middle,
		axis line style={-Stealth,very thick},
		xmin=-10.5,xmax=10.5,ymin=-10.5,ymax=10.5,
		xtick distance=1,
		ytick distance=1,
		extra x ticks={0},
		extra x tick style={grid style={black},xticklabel style={below left=-1.2em},xtick style={black}},
		xticklabel style={font=\tiny, below =-0.6em},
		yticklabel style={font=\tiny, left =-0.6em},
		xlabel=$x$,
		ylabel=$y$,
		xlabel style={right=-0.2cm},
		ylabel style={above = -0.2cm},
		grid=major,
		grid style={thin,densely dotted,black!20}
}}

\makeatletter

\define@key{myAxis}{xmin}{\def\myxmin{#1}}
\define@key{myAxis}{xmax}{\def\myxmax{#1}}
\define@key{myAxis}{ymin}{\def\myymin{#1}}
\define@key{myAxis}{ymax}{\def\myymax{#1}}
\define@key{myAxis}{xlabel}{\def\myxlabel{#1}}
\define@key{myAxis}{ylabel}{\def\myylabel{#1}}
\define@key{myAxis}{scale}{\def\myscale{#1}}
\define@key{myAxis}{addstyle}{\def\myaddstyle{#1}}

\setkeys{myAxis}{xmin=-10.5,ymin=-10.5,xmax=10.5,ymax=10.5,xlabel=$ x $,ylabel=$ y $,scale=0.8,addstyle={}}

\makeatother



\newenvironment{myAxis}[1][]{
	\setkeys{myAxis}{#1}
	\begin{center}
		\begin{tikzpicture}
		\begin{axis}[
		unit vector ratio*=1 1 1,
		width = \myscale\textwidth,
		standardaxis,
		xmin=\myxmin,
		ymin=\myymin,
		xmax=\myxmax,
		ymax=\myymax,
		xlabel=\myxlabel,
		ylabel=\myylabel,
		\myaddstyle]
	}{
		\end{axis}
		\end{tikzpicture}
	\end{center}
}



%---------------------------- fractions

\makeatletter
\define@key{myFraxx}{scale}{\def\myFraxxscale{#1}}
\define@key{myFraxx}{color}{\def\mycolor{#1}}
\define@key{myFraxx}{opacity}{\def\myopacity{#1}}
\define@key{myFraxx}{offset}{\def\myoffset{#1}}
\makeatother

\setkeys{myFraxx}{color=black,opacity=30,scale=1,offset=0}

\newcommand{\fracPie}[3][1]{
	\setkeys{myFraxx}{#1}
	\begin{tikzpicture}[rounded corners=0]
		\def\radius{\myFraxxscale*1cm}
		\def\angl{360/#3}
		\filldraw[fill=\mycolor!\myopacity] (\angl+\angl*\myoffset:\radius) arc (\angl+\angl*\myoffset:(#2+1)*\angl+\angl*\myoffset:\radius) -- (0,0) -- cycle;
		\foreach \n in {1,...,#3}{
			\draw (0,0) -- (\n*\angl:\radius);
		}
		\draw (0,0) circle (\radius);
	\end{tikzpicture}
	\setkeys{myFraxx}{color=black,opacity=30,scale=1,offset=0}
}

\newcommand{\fracPieOutline}[3][1]{
	\setkeys{myFraxx}{#1}
	\begin{tikzpicture}[rounded corners=0]
	\def\radius{\myFraxxscale*1cm}
	\def\angl{360/#3}
	\filldraw[fill=\mycolor!\myopacity] (\angl+\angl*\myoffset:\radius) arc (\angl+\angl*\myoffset:(#2+1)*\angl+\angl*\myoffset:\radius) -- (0,0) -- cycle;
	\phantom{\draw (0,0) circle (\radius);}
	\end{tikzpicture}
	\setkeys{myFraxx}{color=black,opacity=30,scale=1,offset=0}
}

\newcommand{\fracRectH}[3][1]{
	\setkeys{myFraxx}{#1}
	\begin{tikzpicture}[rounded corners=0]
		\def\long{\myFraxxscale*3cm/#3}
		\filldraw[fill=\mycolor!\myopacity] (0,0) rectangle (\long*#2,\long);
		\foreach \n in {1,...,#3}{
			\draw (\n*\long,0) -- (\n*\long,\long);
		}
		\draw (0,0) rectangle (\long*#3,\long);
	\end{tikzpicture}
	\setkeys{myFraxx}{color=black,opacity=30,scale=1,offset=0}
}

\newcommand{\fracRectV}[3][1]{
	\setkeys{myFraxx}{#1}
	\begin{tikzpicture}[rounded corners=0]
	\def\long{\myFraxxscale*3cm/#3}
	\filldraw[fill=\mycolor!\myopacity] (0,0) rectangle (\long,\long*#2);
	\foreach \n in {1,...,#3}{
		\draw (0,\n*\long) -- (\long,\n*\long);
	}
	\draw (0,0) rectangle (\long,\long*#3);
	\end{tikzpicture}
	\setkeys{myFraxx}{color=black,opacity=30,scale=1,offset=0}
}


\newcounter{tmpcounter}
\newcommand{\fracRectTwoD}[3][1]{
	\setkeys{myFraxx}{#1}
	\begin{tikzpicture}[rounded corners=0]
		\def\long{\myFraxxscale*3cm/#3}
		\setcounter{tmpcounter}{0}
		\foreach \i in {#2}{
			\edef\counter{\arabic{tmpcounter}}
			\filldraw[fill=\mycolor!\myopacity] (\counter*\long,0) rectangle (\counter*\long+\long,\long*\i);
			\foreach \n in {1,...,#3}{
				\draw (\counter*\long,\n*\long) -- (\counter*\long+\long,\n*\long);
			}
			\draw (\counter*\long,0) rectangle (\counter*\long+\long,\long*#3);
			\stepcounter{tmpcounter}
		}
	\end{tikzpicture}
	\setkeys{myFraxx}{color=black,opacity=30,scale=1,offset=0}
}


\newcommand{\emptyfrac}{$ \dfrac{\ \blankspace\ }{\ \blankspace\ } $}
