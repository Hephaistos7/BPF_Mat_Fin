%\documentclass[a4paper, 11pt, oneside, BCOR=0mm, DIV=15]{scrbook}

\usepackage{lmodern}
\usepackage[T1]{fontenc}
\usepackage[utf8]{inputenc}
%\usepackage[latin1]{inputenc}
\usepackage[french]{babel}
\usepackage{scrlayer-scrpage}
\usepackage{amsthm}
\usepackage{amsmath}
\usepackage{amssymb}
\usepackage{amsfonts}
\usepackage{latexsym}
\usepackage[calcwidth]{titlesec}
\usepackage[titles]{tocloft}
\usepackage{color}
\usepackage{setspace}
\usepackage{framed}
\usepackage{graphicx}
\usepackage{enumitem}
\usepackage{mdwlist}
\usepackage{multicol, lipsum}
\usepackage{eurosym}
\usepackage{ifthen}
\usepackage{tkz-euclide}
\usepackage{tikz}
\usepackage{tabularx}
\usepackage{array}
\usepackage[pdfborder={0 0 0}]{hyperref}
\usepackage{stmaryrd}
\usepackage{mathrsfs}
\usepackage{pdflscape}
%\usepackage[thinlines]{easytable}
%\usepackage{pst-plot}
%\usepackage{mathtools}
\usepackage{multirow}

%---------------------Jeff
\usepackage{textcomp}
\usepackage{tasks}
\usepackage{ragged2e}
\usepackage{xargs}
\usepackage{pgfplots}
\pgfplotsset{compat=1.11}

\usepackage{icomma}
\usepackage{systeme}
\sysautonum{(*)}

\usetikzlibrary{arrows, shadows, positioning}
\usetikzlibrary{external}
\usetikzlibrary{calc}
\usetikzlibrary{shapes.geometric}
\usetikzlibrary{fit}
\usetikzlibrary{arrows.meta}


% Faarwen défineiren
\def\scriptcolor{navy}
\definecolor{webgreen}{rgb}{0, 0.5, 0} % less intense green
\definecolor{webblue}{rgb}{0, 0, 0.5} % less intense blue
\definecolor{webred}{rgb}{0.5, 0, 0} % less intense red                                                        
\definecolor{lightslategray}{RGB}{119, 136, 153}
\definecolor{darkslategray}{RGB}{47, 79, 79}
\definecolor{steelblue}{RGB}{70, 130, 180}
%\definecolor{midnightblue}{RGB}{25, 25, 112}
\definecolor{darkcyan}{RGB}{0, 139, 139}
%\definecolor{lightseagreen}{RGB}{32,  178, 170}
%\definecolor{aqua}{RGB}{0, 255, 255}
\definecolor{turquoise}{RGB}{0, 206, 209}
%\definecolor{cadetblue}{RGB}{95, 158, 160}
\definecolor{blueviolet}{RGB}{138, 43, 226}
%\definecolor{darkviolet}{RGB}{148, 0, 211}
\definecolor{navy}{RGB}{0, 0, 128}
\definecolor{royalblue}{RGB}{65, 105, 225}
\definecolor{skyblue}{RGB}{135, 206, 250}
\definecolor{slateblue}{RGB}{106, 90, 205}
\definecolor{darkslateblue}{RGB}{72, 61, 139}
\definecolor{palegreen}{RGB}{152, 251, 152}
\definecolor{limegreen}{RGB}{50, 205, 50}
\definecolor{medseagreen}{RGB}{60, 179, 113}
\definecolor{darkgreen}{RGB}{0, 100, 0}
\definecolor{forestgreen}{RGB}{34, 139, 34}
\definecolor{olivedrab}{RGB}{107, 142, 35}
\definecolor{darkolivegreen}{RGB}{85, 107, 47}
\definecolor{orangered}{RGB}{255, 69, 0}
\definecolor{darkorange}{RGB}{255, 140, 0}
\definecolor{indianred}{RGB}{255, 50, 50}
\definecolor{darkred}{RGB}{139, 0, 0}
\definecolor{newdarkred}{RGB}{204, 0, 0}
\definecolor{firebrick}{RGB}{178, 34, 34}
\definecolor{gold}{RGB}{255, 215, 0}
\definecolor{darkgoldenrod}{RGB}{184, 134, 11}
\definecolor{sienna}{RGB}{160, 82, 45}
\definecolor{chocolate}{RGB}{210, 105, 30}
\definecolor{brown}{RGB}{165, 42, 42}
\definecolor{gro}{rgb}{0.75,0.75,0.75}

\def\defcolorborder{firebrick}
\def\defcolorfill{red}
\def\thmcolorborder{darkgoldenrod}
\def\thmcolorfill{gold}
\def\propcolorborder{forestgreen}
\def\propcolorfill{limegreen}


\def\activitycolorborder{chocolate}
\def\activitycolorfill{brown}

\def\exemplecolorborder{navy}
\def\exemplecolorfill{webblue}



\newcommand{\red}{\color{red}}
\newcommand{\green}{\color{forestgreen}}
\newcommand{\blue}{\color{navy}}
\newcommand{\orange}{\color{orange}}
\newcommand{\gro}{\color{gro}}

\everymath{\displaystyle}
%\parskip=2mm
\parindent=0mm

%---------------------- table of contents komplett iwwerschafft
\setcounter{chapter}{0}
\newcommand*\toccolor{%
    \ifcase\value{chapter}%
          navy%----- 0 -- table of contents
    \or   navy%
    \or   darkslateblue%---- Kapitel 1
    \or   steelblue%------ Kapitel 2
    \or   darkgreen%------ Kapitel 3
    \or   darkolivegreen%------ Kapitel 4
    \or   darkgoldenrod%---- Kapitel 5
    \or   sienna%---- Kapitel 6
    \or   firebrick%---- Kapitel 7
    \else navy%-- default
    \fi}

\tocloftpagestyle{scrheadings}



\makeatletter
%% Gliederungsnummer
\renewcommand{\numberline}[1]{%
  \makebox[.8cm][l]{#1}\hspace{1ex}}

\newcommand*\dottedtocline{\leavevmode%
  \leaders\hbox{$\m@th
  \mkern \@dotsep mu\hbox{.}\mkern \@dotsep
  mu$}\hfill\kern\z@} 


\renewcommand{\l@chapter}[2]{%
  \addvspace{1em}%                      vert. Abstand
  \pagebreak[3]%                        Seitenumbruch hier erlauben
  \noindent%                            nicht einrücken
  {\stepcounter{chapter}\color{\toccolor}\bfseries%
%   \Roman{chapter}%
%   \hspace{1ex}%
   \rmfamily #1%
   }%
   \hfill%
   {\color{\toccolor}\bfseries #2}%          Text +  Nummer
  \par%                                 Zeilenumbruch
  \nopagebreak%                         Seitenumbruch nicht erlauben
  \def\scriptcolor{navy}%               Head- & Footline Faarw vum Inhaltsverzeichnis
}
% section
\renewcommand{\l@section}[2]{%
  \noindent\hspace{1cm}%                hor. Einrücken (1cm)
  \rmfamily #1\dottedtocline#2%                           Text + Nummer
  \par%                                 Zeilenumbruch
  \nopagebreak[2]%                      möglichst kein Seitenumbruch
}

%% subsection
\renewcommand{\l@subsection}[2]{%
  \noindent\hspace{2cm}%                hor. Einrücken (2cm)
  \rmfamily #1\dottedtocline#2%                           Text + Nummer
  \par%                                 Zeilenumbruch
}
%% subsubsection
%\renewcommand{\l@subsubsection}[2]{%
%  \noindent\hspace{3cm}%                hor. Einrücken (3cm)
%  \rmfamily #1\dottedtocline#2%                           Text + Nummer
%  \par%                                 Zeilenumbruch
%}
\makeatother

%\renewcommand*\cftchapfont{\stepcounter{chapter}\color{\toccolor}\bfseries}
%\renewcommand*\cftchappagefont{\color{\toccolor}\bfseries}


\newcommand{\toc}
{\pagenumbering{roman}
 \tableofcontents
 \clearpage
 \setcounter{chapter}{0}
 \setcounter{page}{1}
 \pagenumbering{arabic}
 }





%---------------------- kleng Iwwerschreft-Format
\newcommand{\stitle}[1]
 {\medskip
  \noindent{\large\color{\scriptcolor}\bfseries #1}
  \newline
  }



%---------------------- grouss Iwwerschreft-Format
\newlength\netlength
\newcommand{\btitle}[1]
 {\netlength=\textwidth
  \addtolength{\netlength}{-4pt}
  \noindent
  \begin{tikzpicture} % right color=\scriptcolor!50, color=\scriptcolor
  \node[right color=white, left color=\scriptcolor!10, color=\scriptcolor, rectangle, inner sep=2pt] (box)
   {\begin{minipage}{\netlength\scshape\large}
    #1
    \end{minipage}
    };
  \end{tikzpicture}
  }
\newcommand{\btitleinv}[1]
 {\netlength=\textwidth
  \addtolength{\netlength}{-4pt}
  \noindent
  \begin{tikzpicture}
  \node[left color=\scriptcolor!50, right color=white, color=white, rectangle, inner sep=2pt] (box)
   {\begin{minipage}{\netlength\scshape\large}
    #1
    \end{minipage}
    };
  \end{tikzpicture}
  }
  


%---------------------- Layout an de Rand berechnen
\newlength{\rand}
\setlength{\rand}{.5\paperwidth}   % En halwen huelen dat een zum Schluss
\addtolength{\rand}{-0mm}          % BCOR-Wert!!
\addtolength{\rand}{-.5\textwidth} % de Rand vun enger Säit huet.



%---------------------- Headline/Footline
\setheadwidth{\textwidth}
\setheadsepline{1pt}[\color{\scriptcolor}]

\rohead{\normalfont\bfseries \ifthenelse{\equal{\chaptertitle}\empty}{\rightmark}{\chaptertitle}\ %
        {\color{\scriptcolor}\rule[-3mm]{1pt}{7mm}}%
        \rlap{\enskip\thepage}%
        }
\lohead{\classe}
%\scshape\footnotesize FISAR

\chead{}
%\scshape\footnotesize  version \the\day.\the\month.\the\year

\lehead{\llap{\thepage\enskip}{\color{\scriptcolor}\rule[-2mm]{1pt}{7mm}} \normalfont\bfseries \chaptertitle}
\rehead{\classe}

\setfootsepline{.5pt}[\color{\scriptcolor}]


\def\classe{}
\def\lycee{}
%\def\season{2025/26}
%\def\init{JB}

%\ifoot{\small \classe}
%\cfoot{\normalfont\small \lycee}
%\ofoot{\small \season}

\ifoot{\scshape \init}
\cfoot{\normalfont \lycee}
\ofoot{{\scriptsize( v.\the\day.\the\month.)} \small\season}


\pagestyle{scrheadings}
\renewcommand{\chapterpagestyle}{scrheadings}



%---------------------- Kapitel-Iwwerschreften
\newlength{\chapterline}

\titleformat{\chapter}
  [block] % shape
  {% format
   \LARGE\scshape\color{\toccolor}
   }
  {% label
   \hspace{2cm}\thechapter
   }
  {% sep
   1em}
  {% before-code
   }
  [% after-code
   \vspace{-\baselineskip}%
   \hspace{-\rand}%
   \setlength{\chapterline}{\titlewidth}%
   \addtolength{\chapterline}{\rand}%
   \raisebox{-0.1cm}
   {\begin{tikzpicture}[x=1cm, y=1cm]
      \shade[left color = \scriptcolor] (0,0) rectangle (\chapterline,0.05);
    \end{tikzpicture}
    }
   ]
   

\titleformat{\section}
  {\Large\bfseries\color{\scriptcolor}} % font
  {\thesection}{1em}
  {\btitle} % before-code, e.g. TikZ box opening

\titleformat{\subsection}
  {\large\bfseries\color{\scriptcolor}} % font
  {\thesection}{1em}
  {\btitle} % before-code, e.g. TikZ box opening

\titleformat{\subsubsection}
  {\large\bfseries\color{\scriptcolor}} % font
  {\thesection}{1em}
  {\btitle} % before-code, e.g. TikZ box opening


\def\chaptertitle{}

\let\oldchap=\chapter
\renewcommand*{\chapter}
 {%
  \clearpage\def\scriptcolor{\toccolor}%
  \secdef{\Chap}{\ChapS}%
  }
  \newcommand\ChapS[1]
   {\begin{singlespace}
   		\oldchap*{#1}
   		\begin{doublespace}
   			\def\chaptertitle{#1}
   		\end{doublespace}
   	\end{singlespace}
    }
  \newcommand\Chap[2][]
   {\begin{singlespace}
   		\oldchap[#1]{#2}
   		\begin{doublespace}
   			\def\chaptertitle{#2}
   		\end{doublespace}
   	\end{singlespace}
    }

%\newcommand\ChapS[1]            %Jeff: This messes up any linespreading defined by the user in the main document...
%{\singlespacing\oldchap*{#1}\doublespacing
%	\def\chaptertitle{#1}
%	\setstretch{1.2}
%}
%\newcommand\Chap[2][]
%{\singlespacing\oldchap[#1]{#2}\doublespacing
%	\def\chaptertitle{#2}
%	\setstretch{1.2}
%}




%---------------------- Nummereirungen am Rand
\newcommand{\gradstyle}[1]
{\tikzstyle{boxex}=[left color=\scriptcolor,%
                    right color=white,%
                    thick,%
                    rectangle,%
                    inner sep=2pt]%
 \raisebox{-2pt}%
 {\begin{tikzpicture}%
  \node[boxex] (box)%
  {\begin{minipage}{.95\rand\color{\scriptcolor}}%
   \hfill #1%
   \end{minipage}%
   };%
  \end{tikzpicture}%
  }%
 }

\renewcommand*{\othersectionlevelsformat}[1]
 {\llap%
  {\gradstyle%
    {\csname the#1\endcsname}%
    }
  }

\renewcommand*{\thechapter}{\Roman{chapter}}
\renewcommand*{\thesection}{\arabic{section}}

%\addtokomafont{part}{\LARGE\normalfont\scshape\color{\scriptcolor}}
%\addtokomafont{chapter}{\LARGE\normalfont\scshape\color{\scriptcolor}}
%\addtokomafont{section}{\Large\bfseries\color{\scriptcolor}}
%\addtokomafont{subsection}{\large\bfseries\color{\scriptcolor}}
%\addtokomafont{subsubsection}{\large\bfseries\color{\scriptcolor}}



%---------------------- Keschten fir Définitiounen
\newcounter{defcounter}[chapter]

\tikzstyle{dboxstyle} = [fill=\defcolorfill!5,
                         %top color=\defcolorfill!0,
                         %bottom color=\defcolorfill!30,
                         draw=\defcolorfill,
                         %thick,
                         rounded corners,
    rectangle, inner sep=10pt, inner ysep=10pt]

\newcommand{\defn}[2][]
{\stepcounter{defcounter}
 \setlength{\netlength}{\textwidth}
 \addtolength{\netlength}{-20pt}
 
 \bigskip\noindent
 \begin{tikzpicture}
   \node[dboxstyle] (dbox)
   {\begin{minipage}{\netlength}
    \ifthenelse{\equal{#1}{\empty}}{\smallskip}{\medskip}
    #2
    \end{minipage}
    };
   \node[inner ysep=0pt] at (dbox.north west) [anchor=west, xshift=2mm,  fill=white, text=\defcolorfill, top color=white, bottom color=\defcolorfill!5, rounded corners]
   {\textbf{\large Définition \arabic{defcounter}}
    \ifthenelse{\equal{#1}{\empty}}{}{{\large(#1)}}
    };
 \end{tikzpicture}
 }


%---------------------- Keschten fir A retenir (wichteg)
%\newcounter{defcounter}[chapter]

\tikzstyle{dboxstyle} = [fill=\defcolorfill!5,
%top color=\defcolorfill!0,
%bottom color=\defcolorfill!30,
draw=\defcolorfill,
%thick,
rounded corners,
rectangle, inner sep=10pt, inner ysep=10pt]

\newcommand{\retenir}[2][]
{%\stepcounter{defcounter}
	\setlength{\netlength}{\textwidth}
	\addtolength{\netlength}{-20pt}
	
	\bigskip\noindent
	\begin{tikzpicture}
	\node[dboxstyle] (dbox)
	{\begin{minipage}{\netlength}
		\ifthenelse{\equal{#1}{\empty}}{\smallskip}{\medskip}
		#2
		\end{minipage}
	};
	\node[inner ysep=0pt] at (dbox.north west) [anchor=west, xshift=2mm,  fill=white, text=\defcolorfill, top color=white, bottom color=\defcolorfill!5, rounded corners]
	{\textbf{\large A retenir !}
		\ifthenelse{\equal{#1}{\empty}}{}{{\large(#1)}}
	};
	\end{tikzpicture}
}


%---------------------- Késchten fir Théorèmen
\newcounter{thmcounter}[chapter]

\tikzstyle{tboxstyle} = [fill=\thmcolorfill!5,
                         draw=\thmcolorborder,
                         %top color=\thmcolorfill!0,
                         %bottom color=\thmcolorfill!30,
                         %thick,
    rectangle, rounded corners, inner sep=10pt, inner ysep=10pt]

\newcommand{\thm}[2][]
{\stepcounter{thmcounter}
 \setlength{\netlength}{\textwidth}
 \addtolength{\netlength}{-20pt}
 
 \medskip\noindent
 \begin{tikzpicture}
   \node[tboxstyle] (dbox)
   {\begin{minipage}{\netlength}
    \ifthenelse{\equal{#1}{\empty}}{\smallskip}{\medskip}
    #2
    \end{minipage}
    };
   \node at (dbox.north west) [xshift=2mm, anchor=west, fill=white, text=\thmcolorborder, top color=white, bottom color=\thmcolorfill!5, rounded corners]
   {\textbf{\large Théorème \arabic{thmcounter}}
    \ifthenelse{\equal{#1}{\empty}}{}{{\large(#1)}}
    };
 \end{tikzpicture}
 }



%---------------------- Késchten fir Propriété
\newcounter{propcounter}[chapter]

\tikzstyle{pboxstyle} = [fill=\propcolorfill!5,
                         draw=\propcolorborder,
                         %thick, 
                         rounded corners,
                         %top color=\propcolorfill!0,
                         %bottom color=\propcolorfill!30,
    rectangle, inner sep=10pt, inner ysep=10pt]

\newcommand{\prop}[2][]
{\stepcounter{propcounter}
 \setlength{\netlength}{\textwidth}
 \addtolength{\netlength}{-20pt}
 
 \medskip\noindent
 \begin{tikzpicture}
   \node[pboxstyle] (dbox)
   {\begin{minipage}{\netlength}
    \ifthenelse{\equal{#1}{\empty}}{\smallskip}{\medskip}
    #2
    \end{minipage}
    };
   \node at (dbox.north west) [xshift=2mm, anchor=west, fill=white, text=\propcolorborder, top color=white, bottom color=\propcolorfill!5, rounded corners]
   {\textbf{\large Propriété \arabic{propcounter}}
    \ifthenelse{\equal{#1}{\empty}}{}{{\large(#1)}}
    };
 \end{tikzpicture}
 }


%---------------------- Késchten fir Exemple
\newcounter{exemplecounter}[chapter]

\tikzstyle{eboxstyle} = [fill=\exemplecolorfill!5,
draw=\exemplecolorborder,
%thick, 
rounded corners,
%top color=\exemplecolorfill!0,
%bottom color=\exemplecolorfill!30,
rectangle, inner sep=10pt, inner ysep=10pt]

\newcommand{\exemple}[2][]
{\stepcounter{exemplecounter}
	\setlength{\netlength}{\textwidth}
	\addtolength{\netlength}{-20pt}
	
	\medskip\noindent
	\begin{tikzpicture}
	\node[eboxstyle] (dbox)
	{\begin{minipage}{\netlength}
		\ifthenelse{\equal{#1}{\empty}}{\smallskip}{\medskip}
		#2
		\end{minipage}
	};
	\node at (dbox.north west) [xshift=2mm, anchor=west, fill=white, text=\exemplecolorborder, top color=white, bottom color=\exemplecolorfill!5, rounded corners]
	{\textbf{\large Exemple}
		\ifthenelse{\equal{#1}{\empty}}{}{{\large(#1)}}
	};
	\end{tikzpicture}
}

%---------------------- Késchten fir activite
\newcounter{activitycounter}[chapter]

\tikzstyle{aboxstyle} = [fill=\activitycolorfill!5,
draw=\activitycolorborder,
%thick, 
rounded corners,
%top color=\activitycolorfill!0,
%bottom color=\activitycolorfill!30,
rectangle, inner sep=10pt, inner ysep=10pt]

\newcommand{\activite}[2][]
{\refstepcounter{activitycounter}
	\setlength{\netlength}{\textwidth}
	\addtolength{\netlength}{-20pt}
	
	\medskip\noindent
	\begin{tikzpicture}
	\node[aboxstyle] (dbox)
	{\begin{minipage}{\netlength}
		\ifthenelse{\equal{#1}{\empty}}{\smallskip}{\medskip}
		#2
		\end{minipage}
	};
	\node at (dbox.north west) [xshift=2mm, anchor=west, fill=white, text=\activitycolorborder, top color=white, bottom color=\activitycolorfill!5, rounded corners]
	{\textbf{\large Activité \arabic{activitycounter}}
		\ifthenelse{\equal{#1}{\empty}}{}{{\large(#1)}}
	};
	\end{tikzpicture}
}
 
 %---------------------- Rappels
 \newcommand{\rappel}[1]
 { \settowidth{\netlength}{\textbf{Rappel : }}
 	\addtolength{\netlength}{-\textwidth}
 	\par\noindent
 	\textbf{\color{\scriptcolor}Rappel : }%
 	\begin{minipage}[t]{-\netlength}
 		{#1}
 	\end{minipage}\\
 }

%---------------------- Remarques
\newcommand{\rem}[1]
{ \settowidth{\netlength}{\textbf{Remarque : }}
  \addtolength{\netlength}{-\textwidth}
  \par\noindent
  \textbf{\color{\scriptcolor}Remarque : }%
  \begin{minipage}[t]{-\netlength}
  {#1}
  \end{minipage}\\
 }
%%%-------------------- Exercicen définéiren
\newcounter{exoinline}[chapter]
\makeatletter
\newcommand{\addexo}[2]{%
	% Exercice an sol an array schreiwen
	\ifx\exos\undefined
		\def\exos{{#1}/{#2}}
	\else
		\if\exos0
			\def\exos{{#1}/{#2}}
		\else
			\g@addto@macro\exos{,{#1}/{#2}}
		\fi
	\fi%
}

\newcommand{\addexoInline}[2]{%
  % Exercice an sol an array schreiwen
  \ifx\exos\undefined
    \def\exos{{#1}/{#2}}
  \else
    \if\exos0
      \def\exos{{#1}/{#2}}
    \else
      \g@addto@macro\exos{,{#1}/{#2}}
    \fi
  \fi
  % Exercice direkt printen mat Iwwerschrëft asw.
  \medskip\noindent
  \refstepcounter{exoinline}%
  \setlength{\netlength}{\textwidth}
  \addtolength{\netlength}{-20pt}
  \noindent%
\setenumerate[1]{label=\alph*), ref=\theenumi.\alph*}%
 \begin{tikzpicture}
   \node[%dotted, ultra thin, 
         draw=\scriptcolor, rounded corners, inner sep=10pt] (exobox)
   {\begin{minipage}{.46\netlength}
     \tikzset{every picture/.style={rounded corners=0}}
     \medskip \small #1
    \end{minipage}
    };
   \node at (exobox.north west) [xshift=\parindent, yshift=2pt, anchor=west, fill=white, text=\scriptcolor, rounded corners]
   {Exercice \theexoinline
    };%
 \end{tikzpicture}%
 \hfill
 \begin{tikzpicture}
   \node[draw=\scriptcolor, rounded corners, inner sep=10pt] (exobox)
   {\begin{minipage}{.46\netlength}
     \tikzset{every picture/.style={rounded corners=0}}
     \medskip \small #2
    \end{minipage}
    };
   \node at (exobox.north west) [xshift=\parindent, yshift=2pt, anchor=west, fill=white, text=\scriptcolor, rounded corners]
   {Solution \theexoinline
    };%
 \end{tikzpicture}% 
}

\newcommand{\addexoInlineDomicile}[2]{%
	% Exercice an sol an array schreiwen
	\ifx\exos\undefined
	\def\exos{{#1}/{#2}}
	\else
	\if\exos0
	\def\exos{{#1}/{#2}}
	\else
	\g@addto@macro\exos{,{#1}/{#2}}
	\fi
	\fi
	% Exercice direkt printen mat Iwwerschrëft asw.
	\medskip\noindent
	\refstepcounter{exoinline}%
	\setlength{\netlength}{\textwidth}
	\addtolength{\netlength}{-20pt}
	\noindent%
	\setenumerate[1]{label=\alph*), ref=\theenumi.\alph*}%
	\begin{tikzpicture}
	\node[%dotted, ultra thin, 
	draw=\scriptcolor, rounded corners, inner sep=10pt] (exobox)
	{\begin{minipage}{.46\netlength}
		\tikzset{every picture/.style={rounded corners=0}}
		\medskip \small #1
		\end{minipage}
	};
	\node at (exobox.north west) [xshift=\parindent, yshift=2pt, anchor=west, fill=white, text=\scriptcolor, rounded corners]
	{Exercice \theexoinline \ \ (à domicile)
	};%
	\end{tikzpicture}%
	\hfill
	\begin{tikzpicture}
	\node[draw=\scriptcolor, rounded corners, inner sep=10pt] (exobox)
	{\begin{minipage}{.46\netlength}
		\tikzset{every picture/.style={rounded corners=0}}
		\medskip \small #2
		\end{minipage}
	};
	\node at (exobox.north west) [xshift=\parindent, yshift=2pt, anchor=west, fill=white, text=\scriptcolor, rounded corners]
	{Solution \theexoinline
	};%
	\end{tikzpicture}% 
}
 
\newcommand{\addexoInlineVertical}[2]{%
	% Exercice an sol an array schreiwen
	\ifx\exos\undefined
	\def\exos{{#1}/{#2}}
	\else
	\if\exos0
	\def\exos{{#1}/{#2}}
	\else
	\g@addto@macro\exos{,{#1}/{#2}}
	\fi
	\fi
	% Exercice direkt printen mat Iwwerschrëft asw.
	\medskip\noindent
	\refstepcounter{exoinline}%
	\setlength{\netlength}{\textwidth}
	\addtolength{\netlength}{-20pt}
	\noindent%
	\setenumerate[1]{label=\alph*), ref=\theenumi.\alph*}%
	{\centering
	\begin{tikzpicture}
	\node[%dotted, ultra thin, 
	draw=\scriptcolor, rounded corners, inner sep=10pt] (exobox)
	{\begin{minipage}{\netlength}
		\tikzset{every picture/.style={rounded corners=0}}
		\medskip \small #1
		\end{minipage}
	};
	\node at (exobox.north west) [xshift=\parindent, yshift=2pt, anchor=west, fill=white, text=\scriptcolor, rounded corners]
	{Exercice \theexoinline
	};%
	\end{tikzpicture}\\[0.3cm]
	\centering
	\begin{tikzpicture}
	\node[draw=\scriptcolor, rounded corners, inner sep=10pt] (exobox)
	{\begin{minipage}{\netlength}
		\tikzset{every picture/.style={rounded corners=0}}
		\medskip \small #2
		\end{minipage}
	};
	\node at (exobox.north west) [xshift=\parindent, yshift=2pt, anchor=west, fill=white, text=\scriptcolor, rounded corners]
	{Solution \theexoinline
	};%
	\end{tikzpicture}}% 
	
	\smallskip%
	\setenumerate[1]{label=\arabic*.,ref=\arabic*,itemsep=1mm}%
}


\newcommand{\addexoInlineNoSol}[2]{%
	% Exercice an sol an array schreiwen
	\ifx\exos\undefined
	\def\exos{{#1}/{#2}}
	\else
	\if\exos0
	\def\exos{{#1}/{#2}}
	\else
	\g@addto@macro\exos{,{#1}/{#2}}
	\fi
	\fi
	% Exercice direkt printen mat Iwwerschrëft asw.
	\medskip\noindent
	\refstepcounter{exoinline}%
	\setlength{\netlength}{\textwidth}
	\addtolength{\netlength}{-20pt}
	\noindent%
	\setenumerate[1]{label=\alph*), ref=\theenumi.\alph*}%
	{\centering
	\begin{tikzpicture}
	\node[%dotted, ultra thin, 
	draw=\scriptcolor, rounded corners, inner sep=10pt] (exobox)
	{\begin{minipage}{\netlength}
		\tikzset{every picture/.style={rounded corners=0}}
		\medskip \small #1
		\end{minipage}
	};
	\node at (exobox.north west) [xshift=\parindent, yshift=2pt, anchor=west, fill=white, text=\scriptcolor, rounded corners]
	{Exercice \theexoinline
	};%
	\end{tikzpicture}}
	
	\smallskip%
	\setenumerate[1]{label=\arabic*.,ref=\arabic*,itemsep=1mm}%
}

\makeatother

\newcommand{\resetexos}{\def\exos{0}}
  
%---------------------- Exercices-environment
\newcounter{exo}[chapter]
\tikzstyle{exosty}=[anchor=base,%
                    rectangle,%
                    inner sep=2pt,%
                    color=\scriptcolor,
%                    color=white,
%                    bottom color=\scriptcolor,%
%                    top color=\scriptcolor!50,%
                    bottom color=\scriptcolor!10,%
                    top color=\scriptcolor!10,%
                    minimum width=.75cm]

\newcommand{\kescht}[1]{\tikz[baseline]{\node[exosty] (exo) {\bfseries\textsf{#1}};}}

\newcommand{\exo}
		{\par%
		 \refstepcounter{exo}%
		 \noindent%
		 \tikz[baseline]{\node[exosty] (exo) {\bfseries\textsf{\theexo}};}
		 }




\newenvironment{exercices}[1][Exercices]
 {\phantomsection%
  \addcontentsline{toc}{section}{Exercices}%
  \medskip%
  \btitle{\Large #1}%
  \vspace{-1.75\baselineskip}%
  \begin{multicols}{2}%
  \setenumerate[1]{label=\alph*), ref=\theenumi.\alph*}
  \setenumerate[2]{label=\roman*), ref=\theenumii.\roman*} 
  }
 {\setenumerate[1]{ label=\arabic*. , ref=\arabic*, itemsep=1mm}
  \setenumerate[2]{label=\alph*), ref=\theenumi.\alph*}
  \setenumerate[3]{label=\roman*), ref=\theenumii.\roman*} 
  \end{multicols}%
  \vspace{-.5cm}%
  %\noindent\color{\scriptcolor}\rule[0cm]{\textwidth}{1.5pt}%
  }



%---------------------- Solutions-environment
\tikzstyle{solsty}=[anchor=base,%
                    rectangle,%
                    inner sep=2pt,%
                    color=\scriptcolor,%
                    bottom color=white,%
                    top color=\scriptcolor!30,%
                    minimum width=.75cm]

\newcommand{\sol}
		{\par%
		 \refstepcounter{sol}%
		 \noindent%
		 \tikz[baseline]{\node[solsty] (sol) {\bfseries\textsf{\thesol}};}
		 }

\newcounter{sol}[chapter]
\newenvironment{solutions}[1][Solutions]
 {\phantomsection%
  \addcontentsline{toc}{section}{Solutions}%
  \medskip%
  \btitleinv{\Large #1}%
  \vspace{-1.75\baselineskip}%
  \begin{multicols}{2}%
  \setenumerate[1]{label=\alph*), ref=\theenumi.\alph*}
  \setenumerate[2]{label=\roman*), ref=\theenumii.\roman*}
  }
 {\setenumerate[1]{ label=\arabic*. , ref=\arabic*, itemsep=1mm}
  \setenumerate[2]{label=\alph*), ref=\theenumi.\alph*}
  \setenumerate[3]{label=\roman*), ref=\theenumii.\roman*} 
  \end{multicols}%
  \vspace{-.5cm}%
  \noindent\color{\scriptcolor}\rule[0cm]{\textwidth}{1.5pt}%
  }



%---------------------- Retenons
\newcommand{\ret}[2][]
 {\netlength=\linewidth
  \addtolength{\netlength}{-10pt}
  \noindent
  \begin{tikzpicture}
    \node[fill=gray!20, thick, rectangle, inner sep=5pt] (box)
     {\begin{minipage}{\netlength}
       \textbf{Retenons :}\ifthenelse{\equal{#1}\empty}{}{ (#1)} \\
       #2
      \end{minipage}
      };
  \end{tikzpicture}
  }
  
 

%---------------------- p.ex. Environment
\newenvironment{pex}
 {\settowidth{\netlength}{\textbf{p.ex.:}}
  \addtolength{\netlength}{.25cm}
  \par
  \textbf{p.ex.:}%
  \vspace{-\baselineskip}%
  \begin{addmargin}{\netlength}%{minipage}[t]{-\netlength}
  }
 {\end{addmargin}%{minipage}
  \medskip
  }


%---------------------- exemple-type
\newlength{\leftbarwidth}
\setlength{\leftbarwidth}{.25cm}
\newlength{\leftbarsep}
\setlength{\leftbarsep}{.25cm}


\newcommand*{\leftbarcolorcmd}{\color{\scriptcolor}}% as a command to be more flexible
\colorlet{leftbarcolor}{black}

\renewenvironment{leftbar}{%
    \def\FrameCommand{{\leftbarcolorcmd{\vrule width \leftbarwidth\relax\hspace {\leftbarsep}}}}%
    \MakeFramed {\advance \hsize -\width \FrameRestore }%
}{%
    \endMakeFramed
}

\newenvironment{extype}[1][exemple-type :]
 {\vspace{-.5\baselineskip}%
  \begin{leftbar}%
  \vspace{-.25\baselineskip}%
  \stitle{#1}%
  \noindent
  }
 {\end{leftbar}%
  \vspace{-.5\baselineskip}\noindent}%



%---------------------- Nummeréirung nom System 1.a)
\setenumerate[1]{ label=\arabic*. , ref=\arabic*, itemsep=1mm}
\setenumerate[2]{label=\alph*), ref=\theenumi.\alph*}
\setenumerate[3]{label=\roman*), ref=\theenumii.\roman*} 






%---------------------- Jeff

\setitemize[1]{label=\textasteriskcentered)}
\newcolumntype{P}[1]{>{\centering\arraybackslash}p{#1}}


%---------------------- Highlighting

\newcommand{\highlight}[1]{{\color{steelblue} #1}} %highlight in chapters color
